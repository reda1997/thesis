
\chapter{Introduction générale} 

\label{Introduction} 
Durant ces deux dernières décennies les technologies de l’information ont  énormément évolué et leur domaine d’application est de plus en plus vaste. L’internet est utilisée dans tous les secteurs : industrie, éducation, commerce,...etc. La demande des services Cloud augmente considérablement et les gens utilisent de plus en plus les réseaux sociaux. Malheureusement les réseaux traditionnels ne sont pas adéquats à ces nouvelles technologies de l’information  pour plusieurs raisons, par exemple: leur scalabilité limitée, ils consomment trop de ressources matérielles,  leur structure est complexe, ils sont très coûteux et difficiles à gérer.  Ce qui oblige à concevoir des réseaux évolutifs, fiables et robustes pour faire face à ces problèmes.\\

	Comme solution, la technologie Software Defined Networking a été proposée; son but est de rendre les réseaux programmables par logiciel contrairement aux réseaux traditionnels qui nécessitaient l’intervention humaine  pour configurer chaque équipement à part qui est une opération fastidieuse, prend beaucoup de temps et être exposée aux erreurs de configuration. Ce concept de réseau programmable est devenu possible en introduisant un nouvel équipement dit contrôleur. Ce dernier va être le cerveau du réseau : toute décision liée au routage du trafic réseau, la sécurité, la gestion des équipements est prise par le contrôleur.\\
	
	Cependant cette technologie souffre de nombreux problèmes de sécurité, certains sont hérités de l’environnement réseau tandis que d’autre, sont propre à l’architecture SDN. La nature centralisée de l’architecture SDN est exploitée pour mener des attaques contre le contrôleur, une fois infecté, l’attaquant aura un contrôle total du réseau, il peut reconfigurer les équipements, rediriger le trafic réseau ailleurs, lancer d’autres attaques. Donc, c'est essentiel de bien sécuriser l'architecture SDN, spécialement le contrôleur, pour garantir le bon fonctionnement du réseau.\\
	
	Le travail qui nous a été confié, consiste à concevoir un système de détection des attaques par déni de service (DoS) dans un réseau SDN. L’apprentissage automatique, plus précisément, le Clustering, sera la méthode adoptée pour concevoir ce système. Ce système doit écouter les messages échangés dans le réseau et il doit être capable de détecter toute attaque de type DoS en extrayant les informations des flux qui permettent  de spécifier la nature du flux, dire quoi faire lors de l’occurrence d’une attaque et apprendre de nouveaux motifs. Chaque attaque DoS a un motif précis il peut être  le nombre de paquets envoyés, l’intervalle de temps entre deux paquets consécutifs, la durée de l’attaque, la taille moyenne des paquets, ...etc.\\
\newpage
Afin de bien mener le travail dans le cadre approprié, ce mémoire est structuré comme suit; dans un premier lieu, nous commençons par définir le concept des réseaux programmables par logiciel, voir les avantages et les inconvénients qu'apporte cette nouvelle technologie, nous passerons par la suite à la sécurité des réseaux SDN où nous étudierons en détail l'aspect sécurité des réseaux, nous verrons les majeures contraintes de sécurité qui doivent être respectées pour garantir le bon fonctionnement du SDN, les différents type d’attaques auxquels un environnement SDN est exposé et une étude comparative entre des travaux existants dans le même contexte que notre travail. Nous entamerons ensuite la conception en commençant par l'analyse et la spécification des besoins pour passer à la modélisation du système en décrivant son architecture générale et les différents modules qui le composent. Une fois notre système réalisé, il ne reste qu'à le mettre sous le test pour évaluer ses performances et son efficacité dans un réseau SDN simulé. La dernière section de ce mémoire sera une conclusion générale sur le travail que nous allons faire.\\
