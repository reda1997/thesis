% Chapter 4
\chapter{Solution}

\label{Chapter4} 
le Software-Defined Networking a rapidement émergé comme une technologie prometteuse pour les réseaux futurs et a gagné beaucoup d'attention. Cependant, la nature centralisée du SDN rend le système vulnérable aux attaques par déni de services (DoS), une fois le contôleur compris, tout le réseau cessera de fonctionner. Mais cette centralisation a un avantage, la gestion centralisée des équipements réseau, elle permet d'avoir une vue globale des flux de trafic, ce qui offre un meilleur système de défense contre les attaque DoS.\\

Ce chapitre est didié à la présentation la solution proposée qui un système de détection des attaques DoS dans les réseaux SDN. Nous commençons par les hypothèses, son architecture générale et les différents modules qui le composent.

\section{Hypothèses}
Pour que notre système fonctionne de manière sûre et efficace, les hypothèses suivantes sont à prendre en considération :\\
\begin{itemize}
\item[•] Notre système est responsable de surveiller le réseau pour détecter uniquement les attaques de type DoS.\\
\item[•] On suppose que les liens entre le plan de données et le plan de contrôle sont fiables et dotés d’une bande passante suffisante pour faire circuler le trafic de contrôle nécessaire pour le fonctionnement du système et le protocole utilisé pour la communication entre ces deux plans est OpenFlow.\\
\item[•] Le réseau n’a pas été compromis avant ou durant le déploiement du système.\\
\item[•] Dernière hypothèse, qui est la plus importante. Notre modèle a été traîné sur le Dataset suivant xxxxxx. Tout résultat généré par ce système après son déploiement dépendra de cette Dataset.
\end{itemize}

\section{Architecture générale}


 