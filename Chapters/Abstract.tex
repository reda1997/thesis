\chapter{Résumé}

Software-defined Networking, ou les réseaux définis par logiciel, en tant que technologie émergente, apportent de l’innovation dans la mise en réseau. Avec le découplage du plan de contrôle et du plan de données, le SDN offre une architecture  réseau programmable. En raison des avantages qu'apporte cette architecture, de nombreuses entreprises sont passées de l’architecture réseau traditionnelle à la nouvelle architecture SDN. Cependant, SDN comme une nouvelle technologie a surgi plusieurs questions qui posent un défi à son avenir. La sécurité est l’un des principaux enjeux qui menace le SDN. Les attaques par déni de service sont une forme de menace à laquelle les réseaux SDN sont les plus exposés et les dégâts qui peuvent surgir à l’occurrence de ces attaques sont colossales.\\

\noindent Ce travail vise à apporter une solution pour la détection des attaques par déni de service au sein d’un réseau SDN. Nous avons développé, au cours de ce travail, un système de détection d’attaque DoS composé de deux modules ; un qui capture et analyse le flux de trafic et le deuxième identifie la nature de chaque flux capturé (flux bénin ou attaques) en utilisant une approche de clustering. 