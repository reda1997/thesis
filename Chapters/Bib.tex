\frontmatter
\begin{thebibliography}{9}

\bibitem{} 
Open Networking Foundation. Software-Defined Networking: The New Norm
for Networks. ONF White Paper, April 13, 2012.

\bibitem{} 
Open Data Center Alliance. Open Data Center
Alliance Master Usage Model: Software-Defined
Networking Rev. 2.0. White Paper. 2014.

\bibitem{} 
OpenDaylight, https://www.opendaylight.org/

\bibitem{} 
Ryu, http://osrg.github.io/ryu/

\bibitem{}
N.Gude et al. NOX : Towards an operating system for networks. SIGCOMM compute
Commun Rev Vol. 38.numero 3, p.105-110, juill 2008.

\bibitem{} 
Kreutz, D., et al. “Software-Defined Networking: A Comprehensive Survey.”
Proceedings of the IEEE, January 2015.

\bibitem{} 
D. Kreutz, F. M. Ramos, P. E. Verissimo, C. E. Rothenberg, S. Azodol-
molky, and S. Uhlig, “Software-defined networking: A comprehensive sur-
vey,” Proceedings of the IEEE, vol. 103, no. 1, pp. 14–76, 2015.

\bibitem{}
Georgi A. Ajaeiya Nareg Adalian Imad H. Elhajj Ayman Kayssi Ali Chehab "Flow-Based Intrusion Detection System for SDN", American University of Beirut 2017

\bibitem{}
L. Breiman, “Bagging predictors,” Mach. Learn., vol.
24, no. 2, pp. 123–140, 1996. 

\bibitem{}
ABUBAKAR, Atiku and PRANGGONO, Bernardi. "Machine learning based intrusion detection system for
software defined networks" 2017

\bibitem{}
U.M. Fayyad, G. Piatetsky-Shapiro, and P. Smyth, "knowledge discovery and data mining: Towards a unifying Framework", In Proceeding of the 2 nd International Conference on Knowledge Discovery and Data Mining, Menlo Park, California, 1996, pp. 82-88.

\bibitem{}
M. D. Wilkinson, M. Dumontier, I. J. Aalbersberg, G. Appleton, M. Axton, A. Baak, N. Blomberg, J.-W. Boiten, L. B. da Silva Santos, P. E. Bourne, et al., The FAIR Guiding Principles for scientific data
management and stewardship, Scientific Data 3.

\bibitem{}
H. H. Jazi, H. Gonzalez, N. Stakhanova, A. A. Ghorbani, Detecting HTTP-based application layer DoS attacks on web servers in the presence of sampling, Computer Networks 121 (2017) 25–36.

\bibitem{}
I. Sharafaldin, A. H. Lashkari, A. A. Ghorbani, Toward Generating a New Intrusion Detection Dataset and Intrusion Traffic Characterization, in: International Conference on Information Systems Security and Privacy (ICISSP), 2018, pp. 108–116

\bibitem{}
M. Ring, S. Wunderlich, D. Grüdl, D. Landes, A. Hotho, Flow-based benchmark data sets for intrusion detection, in: European Conference on Cyber Warfare and Security (ECCWS), ACPI, 2017, pp. 361–369.

\bibitem{}
M. Alkasassbeh, G. Al-Naymat, A. Hassanat, M. Almseidin, Detecting Distributed Denial of Service Attacks Using Data Mining Techniques, International Journal of Advanced Computer Science and Applications (IJACSA) 7 (1) (2016) 436–445.

\bibitem{}
Canadian Institute for Cybersecurity, https://www.unb.ca/cic/datasets/ids-2018.html

\bibitem{}
Canadian Institute for Cybersecurity, https://www.unb.ca/cic/datasets/ddos-2019.html

\bibitem{}
M. Kumar, “1.7 Tbps DDoS Attack – Memcached UDP Reflections Set New Record,” Accessed on 2018-04-02. [Online]. Available: https://thehackernews.com/2018/03/ddos-attack-memcached.html

\bibitem{}
M. Jonker, A. King, J. Krupp, C. Rossow, A. Sperotto, and A. Dainotti, “Millions of Targets Under Attack: a Macroscopic Characterization of the DoS Ecosystem,” in 2017 ACM IMC, November 2017.

\bibitem{}
Thomas Lukaseder et al, "An SDN-based Approach For Defending Against Reflective DDoS Attacks". Institute of Distributed Systems, Ulm University, Germany.

\bibitem{}
Y. Kim, W. C. Lau, M. C. Chuah, and H. J. Chao, “Packetscore: a statistics-based packet filtering scheme against distributed denial-of-service attacks,” IEEE TDSC, vol. 3, no. 2, April 2006.

\bibitem{}
L. Haiqin, M.S. Kim,” Real-Time Detection of Stealthy DDoS Attacks Using Time-Series Decomposition.Communications.”
Cape Town, South Africa, 23.-27.5.2010, pg. 1-6.

\bibitem{}
S. Noh, C. Lee, K. Choi, and G. Jung, ‘‘Detecting Distributed Denial of Service (DDoS) attacks through inductive learning, Lecture Notes in Computer Science, vol. 2690, pp. 286---295, 2003.

\bibitem{}
Mininet : An Instant Virtual Network on your Laptop (or other PC) – Mininet. https://mininet.org/

\bibitem{}
https://fr.wikipedia.org/wiki/Scikit-learn

\bibitem{}
https://fr.wikipedia.org/wiki/Pandas

\bibitem{}
Argus, https://openargus.org/

\end{thebibliography}